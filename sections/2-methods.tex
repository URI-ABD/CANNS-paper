\section{Methods}
\label{sec:methods}

In this manuscript, we are primarily concerned with $k$-nearest neighbors search in a finite-dimensional 
vector space. Some of the algorithms we present use a modified version of the $\rho$-nearest neighbors search 
algorithm described in CHESS~\cite{ishaq2019clustered} to conduct $k$-NN search. (TODO: make second sentence less 
vague once we determine exactly what we're doing with sharding and indexing.)

Given a dataset $\textbf{X} = \{x_1 \dots x_n\}$, we define a \emph{point} or \emph{datum} $x_i \in \textbf{X}$ as a singular observation (e.g., the genome of 
an organism, the vector representation of a single image, or any entity on which we can define a \emph{distance function}).

We define a \emph{distance function} $f : (\textbf{X}, \textbf{X}) \mapsto \mathbb{R}^+ \cup \{0\}$ as a function which, 
given two points $x, y \in \textbf{X}$, deterministically returns a non-negative real number. We require that the distance function 
be symmetric ($f(x, y) = f(y, x) \forall x, y \in \textbf{X}$) and that the distance between two points $x$ and $y$ be zero if and only if $x = y$. 
When, in addition to these constraints, that the distance function obeys
the triangle inequality, it is a \emph{distance metric}, and in such cases we can guarantee that our search algorithms have perfect recall. 
Choice of distance function varies by type of data. For example, with electromagnetic spectra, we use both 
Euclidean (L2) and Cosine distances. With biological or string data on the other hand, Hamming and Levenshtein distances are more appropriate.

With these concepts defined, we can now pose the $k$- and $\rho$ nearest neighbors search problems.
Given a query $q$, along with a distance function $f$ defined on a dataset $ \textbf{X}$, $k$-nearest neighbors search aims to: 
for a given $k$, find the $k$ closest points to $q$ in $ \textbf{X}$; that is, find the $k$ points $x \in \textbf{X}$ with the smallest $f(q, x)$.
We also have the $\rho$-nearest neighbors search problem: for a given radius $\rho$, find all points in $\textbf{X}$ that are a distance of $\rho$ 
or less from $q$; that is, all points $x \in \textbf{X}$ such that $f(q, x) \leq \rho$.

Some of our algorithms for $\rho$- and $k$-nearest neighbors search rely on the \emph{local fractal dimension} of some region of the dataset, 
which we define as: 
$$\text{log}_2 \left( \frac{|B_D(q, r_1)|}{|B_D(q, r_2)|} \right)$$
where $B_D(q, r)$ is the set of points contained in a ball of radius $r$ 
centered on a point $q$ in the dataset $D$; in this example, we compute fractal dimension for some radius $r_1$ and a smaller radius $r_2 = \frac{r_1}{2}$.
We stress that this concept of local fractal dimension differs from the \emph{embedding dimension} of a dataset. To illustrate the difference,
consider SDSS's APOGEE dataset, wherein each datum is a nonnegative real-valued vector of length 8,575. Hence, the \emph{embedding dimension} of this dataset is 8,575. 
However, due to physical constraints (namely, the laws that govern stellar fusion and spectral emission lines), the data are constrained to a lower-dimensional 
manifold. The \emph{local fractal dimension} is an approximation of the dimensionality of that lower-dimensional manifold at a given point, for some length scale.
The notion that high-dimensional data collected from constrained generating phenomena typically only occupy a low-dimensional manifold is known as the \emph{manifold hypothesis}.


We define a \emph{cluster} as a set of points with a \emph{center}, a \emph{radius}, and an approximation of the \emph{local fractal dimension}.
The \emph{center} is the geometric median of the set of points in the \emph{cluster}, and so it is a real data point. The \emph{radius} is the
maximum distance from a point in the cluster to its \emph{center}. We estimate \emph{local fractal dimension} at the cluster radius and half
the cluster radius. Each cluster (unless it is a leaf cluster) has two child clusters in much the same way that a node in
a binary tree has two child nodes. We define the \emph{metric entropy} of a data set under a hierarchical clustering scheme as a refinement of [7], where
metric entropy for a given cluster radius $r_c$ was the number of clusters of radius $r_c$ needed to cover all data. Here, we use a binary, divisive clustering 
approach, but with early stopping criteria; clusters with fewer than a threshold number of points are not further split. 
Since we frame the asymptotic complexity of $\rho$-NN search in terms of the number of leaf clusters, the \emph{metric entropy} is best thought of in terms of the number of
leaf clusters. 

Given a Cluster $C$, let $c$ be its center and $r$ be its radius. Our $\rho$- and $k$-NN algorithms make use of the following cluster 
properties:
\begin{itemize}
    \item $\delta = f(q, c)$ is the distance from the query to the cluster center $c$.
    \item $\delta_{max} = \delta + r$ is the distance from the query to the theoretically farthest possible instance in $C$.
    \item $\delta_{min} = \text{max}(0, \delta - r)$ is the distance from the query to the theoretically closest possible instance in $C$.
\end{itemize}

We sometimes also use the above notation with points instead of clusters. For example, we sometimes use $\delta = f(x, q)$ to refer to the distance 
from a point $x$ to the query $q$. Since we can think of a point as having radius 0, for points we have that $\delta_{max} = \delta = \delta_{min}$.


\subsection{Clustering}
\label{subsec:methods:clustering}

We start by building a divisive hierarchical clustering of the dataset with CLAM, using a 
similar recursive procedure as outlined in CHESS~\cite{ishaq2019clustered}, but with the following 
improvements: better selection of poles for partitioning and memoization of useful cluster properties, 
such as local fractal dimension (see Section ADD SECTION: $k$-NN by Repeated $\rho$-NN). 


CLAM assumes the manifold hypothesis. 
In other words, we assume that the dataset is embedded in a $D$-dimensional space, but that the data only occupy 
a $d$-dimensional manifold, where $d \ll D$. 
While we sometimes use Euclidean notions, such as voids and volumes, to talk about geometric and topological 
properties of clusters and of the manifold, CLAM does not rely on such notions; 
they serve merely as a convenient, more intuitive way to talk about the underlying mathematics.


For a cluster with $m$ points, we begin by computing a 
random sample of $\sqrt m$ points. We then compute the geometric median of this sample, which we call the 
cluster's center. We then compute the distance from the center to each point in the sample of $\sqrt m$ points. 
The point $l$ which is furthest from the center is designated the left pole, and the point $r$ which is furthest
from $l$ is designated the right pole. We then partition the cluster into a left child and a right child, where the 
left child contains all points in the cluster which are closer to $l$ than to $r$, and the right child contains all 
points in the cluster which are closer to $r$ than to $l$. Without loss of generality, we assign to the left child 
those points which are equidistant from $l$ and $r$. Starting from a root-cluster containing the entire dataset, we 
repeat this procedure until each leaf contains only one datum or until some other user-specified stopping criterion 
is met.


\begin{algorithm} % enter the algorithm environment
\caption{Partition} % give the algorithm a caption
\label{alg:partition} % and a label for \ref{} commands later in the document
\begin{algorithmic}[1] % enter the algorithmic environment
    \REQUIRE $cluster$
    \STATE $m \leftarrow \lfloor \sqrt{|cluster.points|} \rfloor$
    \STATE $seeds \leftarrow m$ random points from $cluster.points$
    \STATE $c \leftarrow$ geometric median of $seeds$
    \STATE $l \leftarrow \argmax d(c,x) \ \forall \ x \in cluster.points$
    \STATE $r \leftarrow \argmax d(l,x) \ \forall \ x \in cluster.points$
    \STATE $left \leftarrow \{x | x \in cluster.points \land d(l,x) \le d(r,x)\}$
    \STATE $right \leftarrow \{x | x \in cluster.points \land d(r,x) < d(l,x)\}$
    \IF{$|left| > 1$}
        \STATE Partition($left$)
    \ENDIF
    \IF{$|right| > 1$}
        \STATE Partition($right$)
    \ENDIF
\end{algorithmic}
\end{algorithm}

\subsubsection {Guaranteed Decrease in Cluster Radii}
\label{subsubsec:methods:guaranteed-decrease-in-cluster-radii}

TODO: This needs some doctoring. 

We show that cluster radius is guaranteed to decrease after at most $d$ partitions.

Assume that we have a dataset which follows a $d$-dimensional distribution embedded in $\mathcal{D}$-dimensional space.
We can describe this distribution choosing some set of $d$ mutually orthogonal axes.
Let $m$ denote the maximum distance between any pair of points along any axis from any choice of mutually orthogonal axes for our $d$-dimensional distribution.
In the worst case scenario, the distribution of our data is a $d$-sphere and $m$ is the maximum distance between two points along every axis for every choice of axes.
In this scenario, the cluster radius is $r = \frac{m}{2}$.

Given a cluster $\mathcal{C}$, our partition algorithm will choose the left and right poles, 
and the axis defined by those poles will be the axis for partitioning $\mathcal{C}$'s points into two child clusters.
For points in each child cluster, the maximum distance between any pair of points along the \emph{axis of partitioning} will have been reduced to at most $\frac{m}{2}$.
Though this bound holds for points on the axis of partitioning, it does not serve as a bound for the maximum distance between \emph{any} pair of points in $\mathcal{C}$.
To see this, consider the two dimensional scenario where the points in $\mathcal{C}$ are distributed in a circle of radius $r$. Clearly, the maximum possible distance between 
any two points in $\mathcal{C}$ is $2r$, and occurs when the points fall on opposite ends of one of its diameters. Now consider a horizontal diameter of the circle, 
and let $l$ denote the leftmost point on this diameter. Next, consider the orthogonal vertical diameter, and let $x$ denote the uppermost point on this diameter. When we 
partition along the horizontal diameter, $l$ and $x$ will be assigned to the left child (as mentioned in ADD SECTION, we assign equidistant points to the left child). 
This reduces the maximum possible distance between any two points on the axis of partition -- the horizontal diameter -- to $r$. However, by considering the 
right triangle formed by $l$, $x$, and the center of the circle, we can see that the distance between 
$l$ and $x$ is $\sqrt{2}{r}$.


If $d > 1$ then the child clusters will have a maximal chord $m_{child}$ such that $m_{child} \leq m$.
By partitioning the cluster in this way, we have taken one axis and reduced the maximum distance along that axis from $m$ to $\frac{m}{2}$.
However, as per the distribution of data as discussed earlier, there may be other axes along which the maximum pairwise distance is still $m$.
We, thus, recursively partition each child cluster.

With each recursive application of Partition, we consume one axis along which the maximum distance was $m$ and reduce the maximum distance along that axis to $\frac{m}{2}$ in the child clusters.
After $d$ recursive applications of partition, we will have exhausted the $d$ axes with the maximum distance of $m$.
After these $d$ partitions, the points in each child cluster will have a maximum pairwise distance of $\frac{m}{2}$ along any axis.
Thus the radii of those child clusters will be at most $\frac{m}{4}$.

\subsubsection {Complexity}
\label{subsubsec:methods:clustering:complexity}

Clustering: Exact partition with $\mathcal{O}(n^2)$ cost vs approximate partition using $\sqrt{n}$ seeds to achieve $\mathcal{O}(n)$ cost.
Building the exact tree costs $\mathcal{O}(n^2 \log n)$ vs approximate tree for $\mathcal{O}(n \log n)$.


\subsection {Index-Building}
\begin{itemize}
    \item sharded search 
    \item index building 
    \item optimization of number of shards 
\end{itemize}

\subsection{\texorpdfstring{$\rho$}{p}-Nearest Neighbors Search}
\label{subsec:methods:rnn-search}

Given a query $q$ and a search radius $\rho$, find all $x \in X$ s.t. $f(q, x) \leq \rho$.

We conduct ranged-nearest neighbors search similarly as described in CHESS~\cite{ishaq2019clustered}, but 
with some implementation improvements including identification of candidate clusters which are leaves and 
conducting exhaustive search over those leaves.

\begin{algorithm} 
    \caption{$\rho$-NN(\emph{clusters, query, r})} 
    \label{alg:rnn} 
    \begin{algorithmic}[2]
        \REQUIRE $r \geq 0$
        \REQUIRE $clusters \neq \emptyset$
        \STATE $results \leftarrow \emptyset$
        \IF{$clusters.left$}
            \IF{$clusters.left.\delta$ $\leq$ $r$ + $clusters.left.radius$}
                \STATE $\rho$-NN($clusters.left, query, r$)
            \ENDIF
        \ENDIF
        \IF{$clusters.right$}
            \IF{$clusters.right.\delta$ $\leq$ $r$ + $clusters.right.radius$} 
                \STATE $\rho$-NN($clusters.right, query, r$)
            \ENDIF
        \ENDIF
        \IF{$\neg$$clusters.left$ $\land \neg$$clusters.right$}
            \FOR{$p \in clusters.points$}
                \IF{$p.\delta \leq r$}
                    \STATE $results \leftarrow r$
                \ENDIF
            \ENDFOR
        \ENDIF
        \STATE Return $results$
    \end{algorithmic}
    \end{algorithm}

The asymptotic complexity of $\rho$-nearest neighbors is the same as in CHESS~\cite{ishaq2019clustered}.

\subsection{\texorpdfstring{$k$}{k}-Nearest Neighbors Search}
\label{subsec:methods:knn-search}

In this section, we present four novel variants of $k$-nearest neighbors search: Greedy Search, K-NN by Repeated $\rho$-NN, Sieve Search I, and Sieve Search II. 
We use a preliminary auto-tuning function (TODO: do we though?) to predict the variant which will perform 
best for a given query, dataset, and value of $k$. We then proceed with search using that variant.  

\subsubsection{Greedy Search}
\label{subsubsec:methods:knn-search:greedy-search}
Greedy Search repeatedly partitions the cluster with the lowest $\delta_{min}$ into its children 
until it is a leaf, removing it and adding its points to a fixed size priority queue of points. Search terminates 
when the furthest hit in the queue of points has $\delta$ greater than the $\delta_{min}$ of the closest remaining
cluster.

Formally, let $Q$ be a priority queue of clusters sorted by non-increasing $\delta_{min}$. Initialize $Q$ with the root cluster.
Let $P$ be a second priority queue of points sorted by non-decreasing distance from the query. This queue starts empty.

While $P$ has fewer than $k$ clusters or $\delta$ of the top priority element in $P$ is greater 
than $\delta_{min}$ of the top priority element in $Q$, do the following:
\begin{enumerate}
\item While the top priority element $q$ in $Q$ isn't a leaf, replace $q$ with its children.
\item Remove $q$ from $Q$ and add all of its points to $P$. 
\item Remove points from $P$ until $P$ has $k$ points. 
\end{enumerate}

Return $P$. 

\begin{algorithm} 
\caption{GreedySearch(\emph{tree, query, k})} 
\label{alg:greedy_search} 
\begin{algorithmic}[3]
    \STATE $Q \leftarrow$ priority queue
    \STATE $Q.push(tree.root)$
    \STATE $P \leftarrow$ priority queue
    \WHILE{$|P| < k \lor P.max.\delta > Q.min.\delta_{min}$}
        \WHILE{$!Q.min.isLeaf$}
            \STATE $[l, r] \leftarrow Q.extractMin().children()$
            \STATE $Q.push([l, r])$
        \ENDWHILE
        \STATE $leaf \leftarrow Q.extractMin()$
        \STATE $P.push(leaf)$
        \WHILE{$|P| > k$}
            \STATE $P.extractMax()$
        \ENDWHILE
    \ENDWHILE
    \STATE Return $P$
\end{algorithmic}
\end{algorithm}


\subsubsection{$k$-NN by Repeated $\rho$-NN}
\label{subsubsec:methods:knn-search:repeated-rnn}
Hence the name, with $k$-NN by Repeated $\rho$-NN we repeatedly perform
$\rho$-nearest neighbors search, increasing the search radius until $k$ neighbors
have been found.

Let $P$ be the set of nearest neighbors found thus far.
Search starts with a radius $m$ equal to the radius of the cluster tree divided by
the cardinality of the dataset. We perform $\rho$-NN search with radius $m$. 
If no points are within a distance $m$ of the query, we increase the radius by a factor of 
2 and perform $\rho$-NN search again, repeating until at least one point is found, i.e., 
until $|P| > 0$.

Once $|P| > 0$, we continue to perform $\rho$-NN search, but instead of 
increasing the radius by a factor of 2 on each iteration, we increase it by a factor determined 
by the local fractal dimension of the clusters containing the points found so far. In particular, 
we increase the radius by a factor of 
$$\text{min}\left(2, \frac{k}{|P|^{\frac{1}{\mu}}}\right)$$
where $\mu$ is the mean local fractal dimension (abbreviated lfd in algorithm below) of the clusters containing the points found so far.
Intuitively, the factor by which we increase the radius should be \emph{inversely} proportional to the number of points found so far. 
Additionally, when the local fractal dimension near the query and found neighbors is low, this suggests that the data 
are relatively concentrated in that region; thus, the factor of radius increase should be \emph{directly} proportional to the 
local fractal dimension. Since the local fractal dimension is memoized at clustering time, this poses no additional cost at search time.
Once $|P| >= k$, we sort the points in $P$ by $\delta$ and return the $k$ closest points.

\begin{algorithm} % enter the algorithm environment
    \caption{Repeated$\rho$-NN(\emph{tree, query, k})} % give the algorithm a caption
    \label{alg:knn-by-rnn} % and a label for \ref{} commands later in the document
    \begin{algorithmic}[4] % enter the algorithmic environment
        \STATE $P \leftarrow$ $\emptyset$
        \STATE $m \leftarrow$ $\frac{tree.radius}{tree.cardinality}$
        \WHILE {$|P| = 0$}
            \STATE $P.push(\rho$-NN$(tree, query, m)$)
            \STATE $m \leftarrow 2m$
        \ENDWHILE
        \WHILE {$|P| < k$}
            \STATE $clusters \leftarrow \{ C: \exists p \in P \land p \in C \}$
            \STATE $\mu \leftarrow \frac{1}{|clusters|} \sum_{C \in clusters} C.lfd$
            \STATE $m \leftarrow \text{min}\left(2, \frac{k}{|P|^{\frac{1}{\mu}}}\right)$
            \STATE $P.push(\rho$-NN$(tree, query, m)$)
        \ENDWHILE
        \STATE $P.sort()$
        \STATE Return $P[0..k]$
    \end{algorithmic}
    \end{algorithm}

\subsubsection{Sieve Search I}
\label{subsubsec:methods:knn-search:sieve}
With Sieve Search I, we repeatedly calculate a threshold distance $\tau$ such that no cluster with a $\delta_{min}$ greater than $\tau$ 
can contain one of the $k$ nearest neighbors. Search terminates when we have a threshold containing exactly $k$ points.

We start by letting $Q$ be a vector of clusters and points sorted by non-decreasing $\delta_{max}$ and initializing $Q$ with the root cluster. 
While we haven't found a threshold which guarantees exactly $k$ points, we do the following: 



TODO: Fix this to reflect new algorithm.

\begin{enumerate}
\item Find the cluster $C_{\tau}$ in $Q$ with the smallest $\delta_{max}$ such that no cluster whose $\delta_{min}$ is greater than $C_{\tau}$'s $\delta_{max}$ can contain one of the $k$ nearest neighbors.
\item Let the threshold $\tau$ be the $\delta_{max}$ of $C_{\tau}$.
\item Remove from $Q$ any cluster whose $\delta_{min}$ is greater than $\tau$.
\item Transfer from $Q$ to $P$ all the points in any cluster whose cardinality is less than $k$ or who is a leaf. 
\item Replace all non-leaf clusters in $Q$ with their children. 
\end{enumerate}

At this stage, if any clusters remain in $Q$, we add all their points to $P$. 
Extract $P$. 

\begin{algorithm} % enter the algorithm environment
    \caption{Sieve } % give the algorithm a caption
    \label{alg:sieve_v1} % and a label for \ref{} commands later in the document
    \begin{algorithmic}[2] % enter the algorithmic environment
        \REQUIRE $tree$, $query$, $k$
        \STATE $Q \leftarrow$ priority queue
        \STATE $Q.push(tree.root)$
        \STATE $P \leftarrow$ priority queue
        \WHILE{$|P| < k \lor P.max.\delta > Q.max.\delta_{min}$}
            \WHILE{$!Q.extractMax().isLeaf$}
                \STATE $[l, r] \leftarrow Q.extractMax().children()$
                \STATE $Q.push([l, r])$
            \ENDWHILE
            \STATE $leaf \leftarrow Q.extractMax()$
            \STATE $P.push(leaf)$
            \WHILE{$|P| > k$}
                \STATE $P.extractMax()$
            \ENDWHILE
        \ENDWHILE
        \STATE Return $P$
    \end{algorithmic}
    \end{algorithm}


\subsubsection{Sieve Search II}
\label{subsubsec:methods:knn-search:sieve2}

\subsubsection{Linear Search}
\label{subsubsec:methods:knn-search:leaf-search}

Let $hits$ be a a fixed size priority queue with $k$ instances sorted by non-decreasing $f(q, x)$, i.e. distance to the query.
Initialize $hits$ with the centers of $candidates$.

Let $f_k$ be the distance to the $k^{th}$ farthest instance so far.
Exclude all $candidates$ for which $\delta^2 > f_k$, i.e. their closest possible instance is farthest that the $k^{th}$ farthest instance found so far.
Remove the closest cluster from $candidates$ and add all its contained instances to $hits$.
Repeat until $candidates$ is empty.

\subsubsection{Complexity}
\label{subsubsec:methods:knn-search:complexity}

This is at-least on-par with the complexity for $\rho$-nearest neighbors search from CHESS.
It is potentially better because we effectively shrink the search radius to adapt to the $k$ closest instances found during the algorithm.


\subsubsection{Leaf Search}
\label{subsubsec:methods:knn-search:leaf-search}

Let $hits$ be a a fixed size priority queue with $k$ instances sorted by non-decreasing $f(q, x)$, i.e. distance to the query.
Initialize $hits$ with the centers of $candidates$.

Let $f_k$ be the distance to the $k^{th}$ farthest instance so far.
Exclude all $candidates$ for which $\delta^2 > f_k$, i.e. their closest possible instance is farthest that the $k^{th}$ farthest instance found so far.
Remove the closest cluster from $candidates$ and add all its contained instances to $hits$.
Repeat until $candidates$ is empty.

\subsubsection{Complexity}
\label{subsubsec:methods:knn-search:complexity}

Clustering: Exact partition with $\mathcal{O}(n^2)$ cost vs approximate partition using $\sqrt{n}$ seeds to achieve $\mathcal{O}(n)$ cost.
Building the exact tree costs $\mathcal{O}(n^2 \log n)$ vs approximate tree for $\mathcal{O}(n \log n)$.

This is at-least on-par with the complexity for $\rho$-nearest neighbors search from CHESS.
It is potentially better because we effectively shrink the search radius to adapt to the $k$ closest instances found during the algorithm.
