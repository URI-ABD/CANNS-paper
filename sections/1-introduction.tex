\section{Introduction}
\label{sec:introduction}
Often dubbed "the Big Data explosion," rapidly advancing technologies have enabled researchers to collect complex data at an unprecedented scale. 
This is especially true in the field of astronomy. 
The Sloan Digital Sky Survey (SDSS)~\cite{blanton2017sdss}, for example, in its APOGEE~\cite{alam2015eleventh} dataset, has collected near-infrared spectra of approximately 430,000 stars in approximately 9,000 dimensions. 
Their MaNGA dataset also contains spectral measurements across the face of each of about 10,000 galaxies across 2,700 deg$^2$. 

Biology has also seen an influx of large, high-dimensional datasets. Silva 18S~\cite{10.1093/nar/gks1219} contains ribosomal DNA sequences of approximately 2.25 
million genomes with aligned length of 50,000 letters; 
the GreenGenes project~\cite{desantis2006greengenes} provides a multiple-sequence alignment of over one million bacterial 16S sequences.


More recently, Big-ANN's NeurIPS'23: Billion Scale Approximate Nearest Neighbor Search Challenge has 
solicits development of data structures and search algorithms for Approximate Nearest Neighbor (ANN) search
on four massive datasets: YFCC 100M~\cite{johnson2019billion} dataset, the Yandex Text-to-Image 10M~\cite{diskann-github}, cross-modal dataset, 
the MSMARCO passage retrieval dataset~\cite{Bruch2023AnAA}, and a 10M slice of the MS Turing data set~\cite{Singh2021FreshDiskANNAF}. These datasets 
are 192, 200, 30,000, and 100 dimensional respectively, and submissions are evaluated with 
a ten million point sample of each. 


In many fields, the rate of data collection outpaces the rate of computing performance improvements predicted by Moore's Law~\cite{brescia2012extracting}, indicating 
that computer architecture will not "catch up" to our computational needs in the near future. As a result, analysis of these large datasets 
requires better algorithms. 

Many researchers are especially interested in performing similarity search on these datasets. 
Similarity search enables a variety of applications, including recommendation and classification systems. 
Additionally, with the rise of large language models such as GPT-3~\cite{2020arXiv200514165B}, similarity search can become a useful tool for 
understanding and verifying the validity of model outputs. 
As the sizes and dimensionalities of datasets have grown, however, efficient and accurate similarity search has become extremely challenging; 
even state-of-the-art algorithms exhibit a steep recall versus speed tradeoff~\cite{ishaq2019clustered}.


In particular, $k$-nearest neighbors ($k$-NN) search is one of the most pervasive classification and recommendation methods in use~\cite{fix1952discriminatory, cover1967nearest}. 
Naive implementations of kNN, whose time complexity is linear in the dataset's cardinality, 
prove prohibitively slow for large datasets. While algorithms for fast $k$-NN on large datasets exist, most of them are approximate and do not 
exploit the geometric and topological structure inherent in large datasets. Approximate search is sufficient for some applications, but the 
need for efficient, \emph{exact} search remains.
In particular, while approximate $k$-NN may agree with exact $k$-NN in a majority vote for large values of $k$, it may be sensitive to local perturbations with smaller values of $k$.

Recent approaches to tackling the exponential growth of data include locality-sensitive hashing~\cite{indyk1999sublinear}, 
clever indexing techniques such as the FM Index~\cite{simpson2010efficient}, and entropy-scaling search~\cite{yu2015entropy, ishaq2019clustered}. 
Entropy-scaling search is a paradigm for similarity search that exploits the geometric and topological structure inherent in big datasets.
Importantly, as suggested by their name, entropy-scaling search algorithms have asymptotic complexity that scales with geometric properties (such as the 
\emph{metric entropy} and \emph{local fractal dimension}, defined in Section 2) of a dataset,
rather than its cardinality. In 2019, we introduced CHESS (Clustered Hierarchical Entropy-Scaling Search), which extended entropy-scaling $\rho$-nearest 
neighbors search to a hierarchical clustering approach. In this paper, we introduce CAKES (CLAM-Accelerated $K$-nearest-neighbor 
Entropy-scaling Search). CLAM (Clustered Learning of Approximate Manifolds), developed to allow ``manifold mapping'' for anomaly detection~\cite{ishaq2021clustered}, is a refinement of the CHESS clustering algorithm. 
Using the cluster tree constructed by CLAM, CAKES builds upon the capabilities of CHESS to perform
$k$-nearest neighbors search. 


This paper focuses on four novel exact $k$-nearest neighbors search algorithms. We also explain some 
implementation improvements to the clustering and $\rho$ nearest neighbors search algorithms in CHESS, 
as well as improved genericity across distance functions. Additionally, we prove a guarantee about the scaling behavior of cluster radii. 
Finally, we provide a comparison of 
CAKES to the state-of-the-art in similarity search, FAISS~\cite{johnson2019billion} on several datasets 
from the ANN and Big ANN benchmarks suites. 