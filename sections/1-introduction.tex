\section{Introduction}
\label{sec:introduction}

Often dubbed ``the Big Data explosion,'' rapidly advancing technologies have enabled researchers to collect complex data at an unprecedented scale. 
% Najib: We should mention datasets in this order: neural -network embeddings, biological, radio-ml, and then astronomical. DONE. (flow is a bit awakward though)


With the rise of large language models such as GPT-3~\cite{2020arXiv200514165B}, GPT-4~\cite{OpenAI2023GPT4TR}, and LLAMA-2~\cite{Touvron2023Llama2O}, neural network embeddings are becoming an important source of big data. 
Biology has also seen an influx of large, high-dimensional datasets.
The GreenGenes project~\cite{desantis2006greengenes} provides a multiple-sequence alignment of over one million bacterial 16S sequences, each 7,682 characters in length;
Silva 18S~\cite{10.1093/nar/gks1219} contains ribosomal DNA sequences of approximately 2.25 million genomes with aligned length of 50,000 letters.


Radio frequency data offers another example of massive datasets. 
For example, the RadioML~\cite{oshea2018radioml} dataset contains 2,555,504 samples of both live and synthetically generated signal captures of different modulation modes over a range of SNR levels.


Additionally, many large, high-dimensional datasets come from the field of astronomy. The Sloan Digital Sky Survey (SDSS)~\cite{blanton2017sdss}, for example, in its APOGEE~\cite{alam2015eleventh} dataset, has collected near-infrared spectra of approximately 430,000 stars in approximately 9,000 dimensions. 


% More recently, Big-ANN's NeurIPS'23: Billion Scale Approximate Nearest Neighbor Search Challenge has 
% solicits development of data structures and search algorithms for Approximate Nearest Neighbor (ANN) search
% on four massive datasets: YFCC 100M~\cite{johnson2019billion} dataset, the Yandex Text-to-Image 10M~\cite{diskann-github}, cross-modal dataset, 
% the MSMARCO passage retrieval dataset~\cite{Bruch2023AnAA}, and a 10M slice of the MS Turing data set~\cite{Singh2021FreshDiskANNAF}. These datasets 
% are 192, 200, 30,000, and 100 dimensional respectively, and submissions are evaluated with 
% a ten million point sample of each. 

In many fields, the sizes of datasets are growing exponentially, and this increase in the rate of data collection outpaces the rate of computing performance improvements predicted by Moore's Law~\cite{brescia2012extracting}.
This indicates that computer architecture will not ``catch up'' to computational needs in the near future.
Thus, analysis of these large datasets requires better algorithms.


Many researchers are especially interested in similarity search on these datasets. 
Similarity search enables a variety of applications, including recommendation~\cite{annoy} and classification systems~\cite{suyanto2022knnclassifier}. 

As the cardinalities and dimensionalities of datasets have grown, however, efficient and accurate similarity search has become extremely challenging; 
even state-of-the-art algorithms exhibit a steep recall versus throughput tradeoff~\cite{ishaq2019clustered}.

We now define two types of similarity search: $k$-nearest neighbor search ($k$-NN) and $\rho$-nearest neighbor search ($\rho$-NN). 
Given a query $q$, along with a distance function $f$ defined on a dataset $\textbf{X}$, $k$-nearest neighbors search 
aims to find the $k$ closest points to $q$ in $ \textbf{X}$.
On the other hand, $\rho$-nearest neighbors search aims to find all points in $\textbf{X}$ that are at most a distance $\rho$ from $q$.


Previous works have used the term \emph{approximate} search to refer to $\rho$-nearest neighbor search, but in this paper, 
we reserve the term \emph{approximate} for search algorithms which do not exhibit perfect recall.
When we refer to an \emph{exact} search algorithm, we mean an algorithm which exhibits perfect recall when compared to a na\"{i}ve linear search.


$k$-nearest neighbors ($k$-NN) search is one of the most ubiquitous classification and recommendation methods in use~\cite{fix1952discriminatory, cover1967nearest}. 
Na\"{i}ve implementations of $k$-NN, whose time complexity is linear in the dataset's cardinality, prove prohibitively slow for large datasets because their cardinalities are growing exponentially.


While algorithms for fast $k$-NN on large datasets do exist, most do not exploit the geometric and topological structure inherent in large datasets.
Further, such algorithms are often approximate and while approximate search may be sufficient for some applications, the need for efficient and \emph{exact} search remains.
For example, for a majority voting classifier, approximate $k$-NN may agree with exact $k$-NN for large values of $k$, but may be sensitive to local perturbations with smaller values of $k$.
This is especially true when classes are not well-separated~\cite{zhang2022imbalanced}.
Further, there is some evidence that distance functions which are non-metrics, such as cosine distance, perform poorly for $k$-NN in biomedical settings~\cite{hu2016distance};
this may suggest that approximate $k$-NN (which may have imperfect precision and recall) could exhibit suboptimal accuracy.

Recent approaches to tackling the exponential growth of data include locality-sensitive hashing~\cite{indyk1999sublinear}, clever indexing techniques such as the FM Index~\cite{simpson2010efficient}, hierarchical navigable small world (HNSW) networks~\cite{Malkov2016EfficientAR}, and entropy-scaling search~\cite{yu2015entropy, ishaq2019clustered}.
Entropy-scaling search is a paradigm for similarity search that exploits the geometric and topological structure inherent in large datasets.
Importantly, as suggested by their name, entropy-scaling search algorithms have asymptotic complexity that scales with geometric properties (such as the \emph{metric entropy} and \emph{local fractal dimension}, defined in Section 2) of a dataset, rather than its cardinality.
In 2019, we introduced CHESS (Clustered Hierarchical Entropy-Scaling Search), which extended entropy-scaling $\rho$-nearest neighbors search to a hierarchical clustering approach.
In this paper, we introduce CAKES (CLAM-Accelerated $K$-nearest-neighbor Entropy-scaling Search).
CLAM (Clustered Learning of Approximate Manifolds), developed to allow ``manifold mapping'' for anomaly detection~\cite{ishaq2021clustered}, is a refinement of the CHESS clustering algorithm. 
Using the cluster tree constructed by CLAM, CAKES builds upon the capabilities of CHESS to perform $k$-nearest neighbors search.

This paper focuses on four novel algorithms for exact $k$-NN search.
We also present some improvements to the clustering and $\rho$-nearest neighbors search algorithms in CHESS, as well as improved genericity across distance functions.
Finally, we provide a comparison of CAKES to the state-of-the-art in similarity search, FAISS~\cite{johnson2019billion}, HNSW~\cite{malkov2016hnsw}, and ANNOY~\cite{annoy} on several datasets from the ANN benchmarks suite~\cite{aumuller2020ann}.
