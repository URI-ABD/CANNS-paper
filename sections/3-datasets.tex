
\section{Datasets And Benchmarking}
\label{sec:datasets-and-benchmarks}

\subsection{ANN Benchmark Datasets}
\label{subsec:ann-benchmarks}


We benchmark on a variety of datasets from the ANN-benchmarks suite~\cite{Aumller2018ANNBenchmarksAB}, along with the appropriate distance function.
Table~\ref{table:datasets:summary} summarizes the properties of these datasets.


The benchmarks for Fashion-Mnist, Glove-25, and Sift (and for their synthetic augmentations) were conducted on an Amazon AWS (EC2) \texttt{r6i.16xlarge} instance, with a 64-core Intel Xeon Platinum 8375C CPU 2.90GHz processor, 512GB RAM.
This is the same configuration used in~\cite{Aumller2018ANNBenchmarksAB}.
The OS kernel was Ubuntu 22.04.3-Ubuntu SMP.
The Rust compiler was Rust 1.72.0, and the Python interpreter version was 3.9.16.

The benchmarks for Silva and RadioML were conducted on an Intel Xeon E5-2690 v4 CPU @ 2.60GHz with 512GB RAM.
The OS kernel was Manjaro Linux 5.15.130-1-MANJARO.
The Rust compiler was Rust 1.72.0, and the Python interpreter version was 3.9.16.


\begin{table}[!t]
    % \renewcommand{\arraystretch}{1.15}
    \caption{Datasets used in benchmarks.}
    \label{table:datasets:summary}
    \vskip 0.15in
    \begin{center}
        \begin{small}
            \begin{sc}
                \begin{tabular}{|l|l|l|l|}
                    \hline
                    \textbf{Dataset} & \textbf{Distance}  &\textbf{Cardinality}  & \textbf{Dimensionality}  \\
                    \hline
                    Fashion-Mnist    & Euclidean              & 60,000             & 784       \\
                    \hline
                    Glove-25         & Cosine                 & 1,183,514          & 25        \\
                    \hline
                    Sift             & Euclidean              & 1,000,000          & 128       \\
                    \hline
                    Random           & Euclidean              & 1,000,000          & 128       \\
                    \hline
                    Silva            & Levenshtein            & 2,224,640          & -         \\
                    \hline
                    RadioML          & Dynamic Time Warping   & 97,920             & 1,024     \\
                    \hline
                \end{tabular}
            \end{sc}
        \end{small}
    \end{center}
    \vskip -0.1in
\end{table}

\subsection{Random Datasets and Synthetic Augmentations}
\label{subsec:random-datasets}

In addition to benchmarks on datasets from the ANN Benchmarks suite, we also benchmarked on synthetic augmentations of these real datasets, using the process described in \ref{subsec:methods:synthetic-data}. 
In particular, we use a noise tolerance $\epsilon = 0.01$ and explore the scaling behavior as the cardinality multiplier (referred to as ``scale'' in Table~\ref{table:results:ann-fashion} {\color{red} TODO: merge the tables into one and confirm this reference is still correct after the merge}) increases. 

We also benchmarked on purely random (i.e., not an augmented version of a known dataset) datasets of various cardinalities. 
For this Random dataset, we used a base cardinality of 1,000,000 and a dimensionality of 128 to match the Sift dataset.
This benchmark allows us to isolate the effect of a manifold structure (which we expect to be absent in a purely random dataset) on the performance of the CAKES' algorithms. 


\subsection{Silva}
\label{subsec:silva}
To showcase the use of CAKES with a more exotic distance function, we also benchmarked on the Silva dataset.
Silva 18S~\cite{10.1093/nar/gks1219} contains ribosomal DNA sequences of 2,224,640 genomes with an aligned length of 50,000 letters.
We use Levenshtein distance on unaligned sequences as the distance function for this dataset.
{\color{red} TODO: Define and/or cite Levenshtein distance.}
% I think this needs another sentence but I'm not sure what else to say


\subsection{Radio ML}
\label{subsec:radioml}
To provide another example of use of CAKES with a more exotic distance function, we benchmarked on the RadioML dataset.
The RadioML dataset contains samples of both live and synthetically generated signal captures of different modulation modes over a range of SNR levels.
Specifically, it is comprised of 24 modulation modes at 26 different SNR levels ranging from -20 dB to 30 dB, with 4,096 samples at each modulation mode and SNR level~\cite{oshea2018radioml}.
Thus, it contains $24 \cdot 26 \cdot 4096 = 2,555,504$ samples in total.
Each sample is a 1,024-dimensional complex-valued vector, representing a signal capture (a time series of complex-valued numbers).
We used a subset of this dataset, containing only the 97,920 samples at 10dB SNR.
We use Dynamic Time Warping as the distance function for this dataset.
{\color{red} TODO: Define and/or cite Dynamic Time Warping distance.}


\subsection{Other Algorithms}
We benchmarked CAKES's algorithms against the state-of-the-art similarity search algorithms HNSW and ANNOY, as well as on FAISS-IVF and two different implementations of linear search.
In particular, we use FAISS-Flat, a Python implementation of linear search, to provide ground-truth values for recall of HNSW, ANNOY, and FAISS-IVF.
We use our own Rust implementation of linear search to provide ground-truth values for recall of CAKES's algorithms.
We plot the throughput of these two implementations of linear search as a reference point for CAKES's algorithms' throughput in Section~\ref{subsec:scaling-behavior-results}.