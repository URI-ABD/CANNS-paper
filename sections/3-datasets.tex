
\section{Datasets And Benchmarking}
\label{sec:datasets-and-benchmarks}

\subsection{ANN Benchmark Datasets}
\label{subsec:ann-benchmarks}


We benchmark on a variety of datasets from the ann-benchmarks suite~\cite{Aumller2018ANNBenchmarksAB}, along with the appropriate distance function.
Table~\ref{table:datasets:summary} summarizes the properties of these datasets.

In addition to benchmarks on datasets from the ANN Benchmarks suite, we also benchmarked on synthetic augmentations of these real datasets, using the process described in \ref{subsec:methods:synthetic-data}. In particular, we use a noise tolerance $\epsilon = 0.01$ and explore the scaling behavior as the multiplier $m$ of new points increases. While we explored the scaling behavior of CAKES with augmented versions of each dataset in \ref{table:datasets:summary}, we give an in-depth discussion in \ref{sec:results} of results for scaling behavior on fashion-mnist, glove-100, gist, and a purely random dataset. 

We compared against FAISS~\cite{johnson2019billion} (IVF and HNSW) and a BLAS-accelerated implementation of na\"ive search as reported in~\cite{johnson2019billion}.

% TODO FIXME is this still true? We used Ark for the scaling benchmarks, no?
All benchmarks were conducted on an Amazon AWS (EC2) \texttt{r6i.16xlarge} instance, with a 64-core Intel Xeon Platinum 8375C CPU 2.90GHz processor, 512GB RAM.
This is the same configuration used in~\cite{Aumller2018ANNBenchmarksAB}
The OS kernel was Ubuntu 22.04.3-Ubuntu SMP. 
The Rust compiler was Rust 1.71.1, and the Python interpreter version was 3.10.12.

\begin{table}[!t]
    % \renewcommand{\arraystretch}{1.15}
    \caption{Datasets used in benchmarks.}
    \label{table:datasets:summary}
    \vskip 0.15in
    \begin{center}
        \begin{small}
            \begin{sc}
                \begin{tabular}{|l|l|l|l|}
                    \hline
                    \textbf{Dataset} & \textbf{Distance}  &\textbf{Cardinality}  & \textbf{Dim.}  \\
                    \hline
                    fashion-mnist    & euclidean              & 60,000             & 784       \\
                    \hline 
                    gist             & euclidean              & 1,000,000          & 960       \\
                    \hline
                    glove-25         & cosine              & 1,183,514          & 25        \\
                    \hline
                    glove-50         & cosine              & 1,183,514          & 50        \\
                    \hline
                    glove-100        & cosine              & 1,183,514          & 100       \\
                    \hline
                    sift             & euclidean              & 1,000,000          & 128       \\
                    \hline
                    deep-image       & cosine              & 9,990,000          & 96        \\
                    \hline
                    mnist            & euclidean              & 60,000             & 784       \\
                    \hline
                    lastfm           & cosine              & 292,385            & 65        \\
                    \hline
                \end{tabular}
            \end{sc}
        \end{small}
    \end{center}
    \vskip -0.1in
\end{table}


\subsection{Silva}
\label{subsec:silva}
Silva 18S~\cite{10.1093/nar/gks1219} contains ribosomal DNA sequences of approximately 2.25 million genomes with an aligned length of 50,000 letters.
We use Levenshtein distance with this dataset.
% I think this needs another sentence but I'm not sure what else to say


\subsection{Radio ML}
The RadioML dataset contains samples of both live and synthetically generated signal captures of different modulation modes over a range of SNR levels. 
Specifically, it is comprised of 24 modulation modes at 26 different SNR levels ranging from -20 dB to 30 dB, with 4,096 samples at each modulation mode and SNR level~\cite{oshea2018radioml}. 
Thus, it contains $24 \cdot 26 \cdot 4096 = 2,555,504$ samples in total. 
Each sample provided in RadioML can be thought of as an array of 2048 floating point values for a single signal capture with labels for SNR and modulation mode.
We use Dynamic Time Warping distance with this dataset. 

\label{subsec:radioml}