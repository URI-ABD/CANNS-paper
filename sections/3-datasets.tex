\section{Datasets And Benchmarks}
\label{sec:datasets-and-benchmarks}

\subsection{ANN-Benchmark Datasets}
\label{sec:datasets-and-benchmarks:ann-benchmark-datasets}

We benchmark on a variety of datasets from the ANN-benchmarks suite~\cite{Aumller2018ANNBenchmarksAB}.
Table~\ref{tab:datasets:summary} summarizes these datasets.
The benchmarks for Fashion-Mnist, Glove-25, Sift, and Random (and for their synthetic augmentations) were conducted on an Amazon AWS (EC2) \texttt{r6i.16xlarge} instance, with a 64-core Intel Xeon Platinum 8375C CPU 2.90GHz processor, 512GB RAM.
This is the same configuration used in~\cite{Aumller2018ANNBenchmarksAB}.
The OS kernel was Ubuntu 22.04.3-Ubuntu SMP.
The Rust compiler was Rust 1.72.0, and the Python interpreter version was 3.9.16.
The benchmarks for SILVA and RadioML were conducted on an Intel Xeon E5-2690 v4 CPU @ 2.60GHz with 512GB RAM.
The OS kernel was Manjaro Linux 5.15.130-1-MANJARO.
The Rust compiler was Rust 1.72.0, and the Python interpreter version was 3.9.16.

\begin{table}[!t]
    % \renewcommand{\arraystretch}{1.15}
    \caption{Datasets used in benchmarks.}
    \label{tab:datasets:summary}
    \vskip 0.15in
    \begin{center}
        \begin{small}
            \begin{sc}
                \begin{tabular}{|l|l|l|l|}
                    \hline
                    \textbf{Dataset} & \textbf{Distance}  &\textbf{Cardinality}  & \textbf{Dimensionality}  \\
                    \hline
                    Fashion-Mnist    & Euclidean              & 60,000             & 784                    \\
                    \hline
                    Glove-25         & Cosine                 & 1,183,514          & 25                     \\
                    \hline
                    Sift             & Euclidean              & 1,000,000          & 128                    \\
                    \hline
                    Random           & Euclidean              & 1,000,000          & 128                    \\
                    \hline
                    SILVA            & Levenshtein            & 2,224,640          & 3,712 - 50,000         \\
                    \hline
                    RadioML          & Dynamic Time Warping   & 97,920             & 1,024                  \\
                    \hline
                \end{tabular}
            \end{sc}
        \end{small}
    \end{center}
    \vskip -0.1in
\end{table}

\subsection{Random Datasets and Synthetic Augmentations}
\label{sec:datasets-and-benchmarks:random-datasets}

In addition to benchmarks on datasets from the ANN-Benchmarks suite, we also benchmarked on synthetic augmentations of these real datasets, using the process described in \ref{sec:methods:synthetic-data}.
In particular, we use a noise tolerance $\epsilon = 0.01$ and explore the scaling behavior as the cardinality multiplier (referred to as ``Multiplier'' in Table~\ref{tab:results:qps-and-recall}) increases.
We also benchmarked on purely randomly-generated datasets of various cardinalities.
For this, we used a base cardinality of 1,000,000 and a dimensionality of 128 to match the Sift dataset; hereafter, we refer to the random dataset with cardinality 1,000,000 and dimensionality 128 as ``Random.''
This benchmark allows us to isolate the effect of a manifold structure (which we expect to be absent in a purely random dataset) on the performance of the CAKES' algorithms.


\subsection{SILVA 18S}
\label{sec:datasets-and-benchmarks:silva-18s}

To demonstrate CAKES with a more exotic distance function, we also benchmarked on the SILVA 18S ribosomal RNA dataset~\cite{10.1093/nar/gks1219}.
This dataset contains ribosomal RNA sequences of 2,224,640 genomes with an aligned length of 50,000 letters.
We held out a set of 1,000 random sequences from the dataset to use as queries for benchmarking.
We use Levenshtein~\cite{levenshtein1966binary} distance on the unaligned sequences to build the tree and to perform $k$-NN search.
Since this dataset contains many redundant, i.e. nearly identical, sequences, we used a maximum tree-depth of 128 as a partition criterion in Algorithm~\ref{alg:methods:partition}.


\subsection{Radio ML}
\label{sec:datasets-and-benchmarks:radio-ml}

As another example of an exotic distance function, we benchmarked CAKES on the RadioML dataset~\cite{oshea2018radioml}.
The RadioML dataset contains samples of synthetically generated signal captures of different modulation modes over a range of SNR levels.
Specifically, it comprises 24 modulation modes at 26 different signal-to-noise ratio (SNR) levels ranging from -20 dB to 30 dB, with 4,096 samples at each combination of modulation mode and SNR level.
Thus, it contains $24 \cdot 26 \cdot 4096 = 2,555,504$ samples in total.
Each sample is a 1,024-dimensional complex-valued vector, representing a signal capture (a time-series of complex-valued numbers).
We used a subset of this dataset, containing 97,920 samples at 10dB SNR and used the other 8,576 samples at 10dB SNR as a hold-out set of queries.
We use Dynamic Time Warping~\cite{muller2007dynamic} as the distance function for this dataset.


\subsection{Other Algorithms}
\label{sec:datasets-and-benchmarks:other-algorithms}

We benchmarked CAKES's algorithms against the state-of-the-art similarity search algorithms HNSW, ANNOY and FAISS-IVF, and two implementations of linear search.
In particular, we use FAISS-Flat, a Python implementation of linear search, to provide ground-truth values for recall of HNSW, ANNOY, and FAISS-IVF.
We use our own Rust implementation of linear search to provide ground-truth values for calculating recall of CAKES's algorithms.
We plot the throughput of these two implementations of linear search as a reference point for all other algorithms in Figure~\ref{fig:results:scaling-plots}.
