\section{Methods}
\label{sec:methods}

Write about methods and provide asymptotic complexity analysis of each method.

\subsection{Partition}
\label{subsec:methods:partition}

Partition algorithm for building the tree.
Lift this from the CHAODA paper.

Exact partition with $\mathcal{O}(n^2)$ cost vs approximate partition using $\sqrt{n}$ seeds to achieve $\mathcal{O}(n)$ cost.
Building the exact tree costs $\mathcal{O}(n^2 \log n)$ vs approximate tree for $\mathcal{O}(n \log n)$.

\subsection{\texorpdfstring{$\rho$}{p}-Nearest Neighbors Search}
\label{subsec:methods:rnn-search}

Given a query $q$ and a search radius $\rho$, find all $x \in X$ s.t. $f(q, x) \leq \rho$.

Modified binary search to identify candidate leaf clusters.
Exhaustive leaf search over those leaves.

Lift this from CHESS paper. Be sure to cite CHESS here.

Complexity is the same as in CHESS.

\subsection{\texorpdfstring{$k$}{k}-Nearest Neighbors Search}
\label{subsec:methods:knn-search}

The k-nearest neighbors search problem is posed as follows: given a query $q$ and a positive integer $k$, find the $k$ closest instances in $X$.

Given a Cluster $C$, let $c$ be its center and $r$ be its radius. Our $k$-nn variants make use of the following definitions:
\begin{itemize}
    \item $\delta = f(q, c)$ is the distance from the query to the cluster center.
    \item $\delta_{max} = \delta + r$ is the distance from the query to the potentially farthest instance in the Cluster.
    \item $\delta_{min} = \text{max}(0, \delta - r)$ is the distance from the query to the potentially closest instance in the Cluster.
\end{itemize}

\subsubsection{Expanding Threshold}
\label{subsubsec:methods:knn-search:expanding-threshold}

Let $candidates$ be a priority queue of clusters sorted by non-increasing $\delta_{min}$. Initialize $candidates$ with the root cluster.
Let $hits$ be a second priority queue of points sorted by non-decreasing distance from the query. This queue starts empty.

While $hits$ has fewer than $k$ clusters or the $\delta$ of the top priority hit is greater 
than the $\delta_{min}$ of the top priority candidate, do the following:
\begin{enumerate}
\item While the top priority candidate isn't a leaf, replace the top priority 
candidate with its children.
\item Remove the top priority candidate from $candidates$ and add all of its 
points to $hits$. 
\item Remove points from $hits$ until $hits$ has $k$ points. 
\end{enumerate}

Extract $hits$. 


\begin{algorithm} % enter the algorithm environment
\caption{Expanding Threshold} % give the algorithm a caption
\label{alg:expanding_threshold} % and a label for \ref{} commands later in the document
\begin{algorithmic}[1] % enter the algorithmic environment
    \REQUIRE $tree$, $query$, $k$
    \STATE $candidates \leftarrow$ priority queue
    \STATE $candidates.push(tree.root)$
    \STATE $hits \leftarrow$ priority queue
    \WHILE{$|hits| < k \lor hits.peek().\delta > candidates.peek().\delta_{min}$}
        \WHILE{$!candidates.peek().isLeaf$}
            \STATE $[l, r] \leftarrow candidates.pop().children()$
            \STATE $candidates.push([l, r])$
        \ENDWHILE
        \STATE $leaf \leftarrow candidates.pop()$
        \STATE $hits.push(leaf)$
        \WHILE{$|hits| > k$}
            \STATE $hits.pop()$
        \ENDWHILE
    \ENDWHILE
    \STATE Return $hits$
\end{algorithmic}
\end{algorithm}


\subsubsection{Sieve V1}
\label{subsubsec:methods:knn-search:sieve-v1}
Let $candidates$ be a vector of $clusters$s sorted by non-decreasing $\delta_{max}$. Initialize $candidates$ with  
the root cluster. 
Let $hits$ be a priority queue of size $k$ containing points sorted by non-decreasing distance from the query. This queue starts empty.

In this variant of KNN, we repeatedly calculate a threshold distance $\tau$ such that no cluster with a $\delta_{minmax}$ less than $\tau$ 
can contain one of the $k$ nearest neighbors. 

Repeat the following: 
\begin{enumerate}
\item Find the cluster $C_{\tau}$ in $candidates$ with the smallest $\delta_{max}$ such that no cluster whose $\delta_{min}$ is greater than $C_{\tau}$'s $\delta_{max}$ can contain one of the $k$ nearest neighbors.
\item Let the threshold $\tau$ be the $\delta_{max}$ of $C_{\tau}$.
\item Remove from $hits$ any points which are a distance greater than $\tau$ from the query. 
\item If no cluster in $candidates$ has a $\delta_{max}$ less than or equal to $\tau$, break.
\item Remove from $candidates$ any cluster whose $\delta_{min}$ is greater than $\tau$.
\item Transfer from $candidates$ to $hits$ all the points in any cluster whose cardinality is less than $k$ or who is a leaf. 
\item Replace all non-leaf clusters in $candidates$ with their children. 
\end{enumerate}

At this stage, if any clusters remain in $candidates$, we add all their points to $hits$. 
Extract $hits$. 

\begin{algorithm} % enter the algorithm environment
    \caption{Sieve V1} % give the algorithm a caption
    \label{alg:sieve_v1} % and a label for \ref{} commands later in the document
    \begin{algorithmic}[2] % enter the algorithmic environment
        \REQUIRE $tree$, $query$, $k$
        \STATE $candidates \leftarrow$ priority queue
        \STATE $candidates.push(tree.root)$
        \STATE $hits \leftarrow$ priority queue
        \WHILE{$|hits| < k \lor hits.peek().\delta > candidates.peek().\delta_{min}$}
            \WHILE{$!candidates.peek().isLeaf$}
                \STATE $[l, r] \leftarrow candidates.pop().children()$
                \STATE $candidates.push([l, r])$
            \ENDWHILE
            \STATE $leaf \leftarrow candidates.pop()$
            \STATE $hits.push(leaf)$
            \WHILE{$|hits| > k$}
                \STATE $hits.pop()$
            \ENDWHILE
        \ENDWHILE
        \STATE Return $hits$
    \end{algorithmic}
    \end{algorithm}

\subsubsection{Sieve V2}
\label{subsubsec:methods:knn-search:sieve-v2}
Add explanation of the sieve v2 algorithm.

\subsubsection{Repeated Rnn}
\label{subsubsec:methods:knn-search:repeated-rnn}
Add explanation of the repeated rnn algorithm.

\subsubsection{Leaf Search}
\label{subsubsec:methods:knn-search:leaf-search}

Let $hits$ be a a fixed size priority queue with $k$ instances sorted by non-decreasing $f(q, x)$, i.e. distance to the query.
Initialize $hits$ with the centers of $candidates$.

Let $f_k$ be the distance to the $k^{th}$ farthest instance so far.
Exclude all $candidates$ for which $\delta^2 > f_k$, i.e. their closest possible instance is farthest that the $k^{th}$ farthest instance found so far.
Remove the closest cluster from $candidates$ and add all its contained instances to $hits$.
Repeat until $candidates$ is empty.

\subsubsection{Complexity}
\label{subsubsec:methods:knn-search:complexity}

This is at-least on-par with the complexity for $\rho$-nearest neighbors search from CHESS.
It is potentially better because we effectively shrink the search radius to adapt to the $k$ closest instances found during the algorithm.

\subsection{Compression}
\label{subsec:methods:compression}

Encode and Decode algorithms.
These could be, and probably will be, different for each $(X, f)$ pairing.

Asymptotic complexity analysis for:
\begin{enumerate}[i.]
    \item encode (novel contribution)
    \item serialize
    \item compress
    \item decompress
    \item deserialize
    \item decode (novel contribution)
\end{enumerate}

% discrete vs continuous, and small alphabet vs large alphabet in discrete
